%!TEX root = ../main.tex
% !TeX spellcheck = en_GB 

% Chapter 1

\chapter{Introduction} % Main chapter title

\label{Introduction} % For referencing the chapter elsewhere, use \ref{Chapter1} 

%----------------------------------------------------------------------------------------

% Define some commands to keep the formatting separated from the content 
\newcommand{\keyword}[1]{\textbf{#1}}
\newcommand{\tabhead}[1]{\textbf{#1}}
\newcommand{\code}[1]{\texttt{#1}}
\newcommand{\file}[1]{\texttt{\bfseries#1}}
\newcommand{\option}[1]{\texttt{\itshape#1}}

%----------------------------------------------------------------------------------------

Maritime shipping can be considered one of the pillars of the modern economy. In fact the \textcite{percent_trade} state that 90\% of the worlds total trade is handled by maritime shipping. This sums up to roughly 1 600 000 seafarers serving on international trading merchant ships worldwide. However, the recent rapid development of sensor technology and artificial intelligence could potentially reduce the operational costs of such vessels, by facilitating the developments of  Unmanned surface vehicles (USVs). The number of persons needed to operate vessels could thereby be reduced significantly. Furthermore, a majority of the accidents reported between 2011 and 2016 can be linked to human erroneous actions \cite{marine_casualities_incidents_2017}. USVs could, thus potentially decrease both the operational costs and the number of accidents.

It is therefore of utmost interest to overcome the challenges connected to the development of USVs. One of these challenges  is the development of algorithms to handle collision avoidance between vessels in agreement with the rules of navigation. Various approaches have already been tried. This thesis will further examine the use of  fuzzy logic in a collision avoidance algorithm for USVs.
\section{Purpose}
The basis for this research lies in the work done by \textcite{perera2012intelligent}. However, this thesis tries to further test previous findings by recreating the previous implementation in python instead of MatLab and testing the algorithm in more challenging simulation scenarios than before.

\section{Disposition}
This first chapter  provides  general information about the thesis along with background information on the topic. Furthermore, it presents the purpose of the thesis. Additional background information on USVs, as well as their advantages and challenges, are presented in chapter \ref{Unmanned_Surface_Vehicles}. Chapter \ref{sec_colreg} explains the International Regulations for Preventing Collisions
at Sea (COLREGs) and describes a few collision avoidance situations.  Previous approaches to automation of COLREGs rules are presented in chapter \ref{chap:aut_colregs} after which the theory for the fuzzy logic approach used in this thesis is thoroughly explained in chapter \ref{chap:fuzzy}. The implementation of the fuzzy logic based Autonomous Navigation System and the simulation framework is described in chapter \ref{chap_impl}. Chapter \ref{sec:evaluation} presents the simulation scenarios as well as results and evaluations of the simulations. The results from chapter \ref{sec:evaluation} are discussed in chapter \ref{chap:disc}. Finally, conclusions are provided in chapter \ref{chap:conc}.